\documentclass[a4paper, 11pt, finnish]{article}
\usepackage{ucs}
\usepackage[utf8x]{inputenc}
\usepackage[T1]{fontenc}
\usepackage[finnish]{babel}

\setlength{\parindent}{0pt}
\setlength{\parskip}{1ex plus 0.5ex minus 0.2ex}

\author{Topi Talvitie}
\title{turrIDEvelop: Tiedostomuoto}

\begin{document}
\maketitle

Yksittäinen Turingin kone tallennetaan seuraavaa muotoa noudattavaan
JSON-tiedostoon:

\textbf{Turingin kone}, juuriolio:
\begin{itemize}
\item states : mappi tilojen nimiltä \textbf{tiloille}
myöhemmin)
\end{itemize}

\textbf{Tila}, olio:
\begin{itemize}
\item transitions : taulukko \textbf{tilasiirtymiä}
\item accepting : onko tila hyväksyvä
\item submachine : tilaan tultaessa ajettava projektin toisen koneen nimi tai
null jos ei ajeta mitään
\item x : x-koordinaatti esityksessä kaaviona
\item y : y-koordinaatti esityksessä kaaviona
\end{itemize}

\textbf{Tilasiirtymä}, olio:
\begin{itemize}
\item destination : kohdetila
\item inchar : merkit joista tilasiirtymä voi lukea yhden.
\item outchar : merkki jonka tilasiirtymä kirjoittaa, jos null niin ei
kirjoita mitään.
\item move : siirtymä kirjoittamisen jälkeen, ''L'' = vasemmalle, ''R'' =
oikealle, ''S'' = pysy paikallaan (''Stay'').
\end{itemize}
\end{document}
