\documentclass[a4paper, 11pt, finnish]{article}
\usepackage{ucs}
\usepackage[utf8x]{inputenc}
\usepackage[T1]{fontenc}
\usepackage[finnish]{babel}

\setlength{\parindent}{0pt}
\setlength{\parskip}{1ex plus 0.5ex minus 0.2ex}

\author{Topi Talvitie}
\title{turrIDEvelop: Aihemäärittely}

\begin{document}
\maketitle
Ohjelma on kehitysympäristö Turing-koneille. Ohjelmassa Turing-koneita
ohjelmoidaan piirtämällä tilakaavioita. Turing-koneisiin sallitaan myös
laajennos, että koneen osana voidaan ajaa muita saman projektin (eli
hakemiston) koneita.

Tässä toteutettavissa Turing-koneissa aakkostona on Javan tukemat
Unicode-merkit (char), syöte on nauhan alussa ja loput merkit ovat
unicode-openbox-merkkejä (␣). Jokaisessa tilasiirtymässä luetaan jokin
annetuista merkeistä, voidaan kirjoittaa tilalle jokin merkki ja liikutaan
nauhalla vasemmalle, oikealle tai pysytään paikallaan. Ohjelman suoritus
päättyy jos se saavuttaa hyväksyväksi merkityn tilan. Mikäli nauhan alussa
yritetään liikkua vasemmalle, jää lukupää samaan kohtaan.

\section*{Toiminnot}
\begin{itemize}
\item Projektin avaaminen, eli hakemiston valitseminen. Hakemistosta avataan
tällöin kaikki hakemiston konetiedostot jotka on tallennettu yksinkertaiseen
JSON-formaattiin.
\item Konetiedoston tallentaminen.
\item Uuden koneen luominen projektiin.
\item Kaikkien projektin konetiedostojen tallentaminen samaan aikaan.
\item Turing-koneen simuloiminen. Suoritus aloitetaan käyttäjän valitseman
koneen 'start'-nimisestä tilasta.
Käyttäjä antaa koneen syötteen joko tiedostosta tai tekstikentästä ja voi
kontrolloida suoritusta joko steppaamalla seuraavaan tilaan, seuraavaan
ajamalla ohjelman läpi. Nauhan tilaa näytetään reaaliajassa.
\item Tilan lisääminen koneeseen.
\item Tilan tietojen muuttaminen. Tilalla voi olla nimi, ja se voidaan merkitä
hyväksyväksi tilaksi. Mikäli tilalla ei ole nimeä, sille
annetaan siinä koneessa minimaalinen uniikki numerotunnus. Tilalle ei voi
antaa nimeä joka on jo toisen tilan käytössä. Tilaan voi merkitä myös
alikoneen nimen, eli aina kun simulaatiossa päästään kyseiseen tilaan käydään
suorittamassa annetun niminen kone alkutilastaan hyväksyvään tilaan asti ja
palataan jatkamaan suoritusta.
\item Tilan poistaminen koneesta. Samalla poistetaan tilaan liittyvät
tilasiirtymät.
\end{itemize}
\end{document}
