\documentclass[a4paper, 11pt, finnish]{article}
\usepackage{graphicx}
\usepackage{ucs}
\usepackage[utf8x]{inputenc}
\usepackage[T1]{fontenc}
\usepackage[finnish]{babel}

\setlength{\parindent}{0pt}
\setlength{\parskip}{1ex plus 0.5ex minus 0.2ex}

\author{Topi Talvitie}
\title{turrIDEvelop: Käyttöohje}

\begin{document}
\maketitle

\section{Projektit}
turrIDEvelop on kehitysympäristö Turingin koneiden luomiseen. Kehitysympäristö
käsittelee projekteja, jotka tallennetaan hakemistoina jotka sisältävät
.turr-päätteellä nimettäviä konetiedostoja (päätteen kirjaimet pienellä).
Muut tiedostot projektikansiossa kehitysympäristö jättää huomiotta.

Projektin avaaminen onnistuu valitsemalla Project-valikosta Open project ja
valitsemalla projektikansion. Nykyinen muokattava projekti tallennetaan
valitsemalla Project-valikosta Save project ja valitsemalla projektikansio.
Jos projektikansiota ei ole olemassa, on se luotava erikseen.

\section{Koneet}
Projektin koneita (joita vastaavat projektihakemiston .turr-tiedostot) voi
muokata ja valita vasemman yläkulman Machines-listasta. Kun kone valitaan,
ikkunan oikealle puolelle aukeaa koneen diagrammiesitys.
Diagrammiesityksessä voidaan siirtää näytettävää kohtaa painamalla oikea
hiirinappi alas ja liikuttamalla hiirtä. Tilojen sijaintia voi myös muuttaa
raahaamalla niitä.

\section{Tilat}
Kun diagrammiesityksestä klikataan tilaa, aukeaa tilan muokkauspaneeli ikkunan
vasempaan alakulmaan. Tässä paneelissa voi vaihtaa koneen nimeä, alikonetta ja
muokata tilasta lähteviä siirtymiä, tai muuttaa tilan hyväksyväksi. Uutta
siirtymää luotaessa on valittava siirtymän kohdetila koneen diagrammiesityksestä.

\section{Alikoneet}
turrIDEvelop tukee Turingin koneeseen tavallisesti kuulumatonta ominaisuutta:
voit luoda tilaan viittauksen projektin toiseen koneeseen. Tällöin kun koneen
simulaatio saapuu tilaan, käy simulaatio ajamassa viitatun alikoneen
'start'-tilasta hyväksyvään tilaan asti ja sitten palaa jatkamaan suoritusta
viittaavasta tilasta. Alikone määritellään tilan muokkauspaneelissa, ja se
piirretään kaavioon tilan alapuolelle suorakulmioon.

\section{Simulointi}
Projektin kone voidaan ajaa valitsemalla Project-valikosta Run tai painamalla
Ctrl+R. Tällöin diagrammiesityksen alareunaan aukeaa paneeli, jossa voi
säädellä koneen ajoa. Run-paneelin alareunassa näkyy simulaatiossa käytettävä
nauha, jota voi editoida. Kone käynnistetään valitsemalla alasvetovalikosta
ajettava kone ja painamalla Start. Ajo aloitetaan aina 'start'-nimisestä
tilasta. Konetta voidaan ajaa joko yksi siirtymä
kerrallaan painamalla Step tai jatkuvasti painamalla Run pohjaan. Oikeassa
reunassa näytetään simulaation nykyinen tilanne. Simulointi suorittaa myös
alikoneet ja saavuttaessa alikoneeseen viittaavaan tilaan simulointi hyppää
suoraan alikoneeseen.

\section{Pikanäppäimet}
Tärkeimpiin käyttöliittymän nappeihin liittyy pikanäppäin, esimerkiksi Alt+N
luo nykyiseen koneeseen uuden tilan ja Alt+T luo nykyiseen tilaan uuden
siirtymän. Pikanäppäimenä käytettävä kirjain on napissa alleviivattuna.

\end{document}
